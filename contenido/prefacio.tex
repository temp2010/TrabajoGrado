\vspace*{3cm}
\noindent\Huge\textsc{Introducción}\\
\normalsize
\noindent\rule[2pt]{\textwidth}{0.8pt}
\hspace*{3cm}

Con el trabajo desarrollado se elaboró una propuesta de Arquitectura empresarial para el producto mistituto.com de la empresa Creatics la cual se dedica al desarrollo de software.  Este proceso se generó a través de la identificación de las características organizacionales y la representación de sus componentes y sus relaciones de forma integral utilizando el lenguaje Archimate\index{Archimate}. \\ \\

La Arquitectura Empresarial\index{Arquitectura Empresarial} es una metodología que describe formalmente el sistema visualizando de forma global los elementos de las organizaciones, su relación en la consecución de los objetivos estratégicos y su evolución en el tiempo.
Archimate\index{Archimate} es un lenguaje de modelamiento de Arquitectura empresarial que mediante un conjunto de símbolos y estructuras gráficas permite representar la arquitectura empresarial. \\ \\

El documento se encuentra dividido en cuatro partes; la primera parte contiene la contextualización, la cual incluye la descripción y filosofía organizacional de la empresa y la conceptualización.  En la segunda parte se desarrolla la arquitectura empresarial. En la tercera parte se presentan las conclusiones y por último la cuarta parte relaciona referencias bibliográficas.