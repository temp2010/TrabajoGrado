\pdfbookmark[0]{Resumen}{Abstract}
\chapter*{Resumen}
\label{sec:resumen}
\vspace*{-10mm}
El objetivo del trabajo de grado es el desarrollo de la arquitectura empresarial para la empresa Creatics y su producto Minstituto. La arquitectura empresarial se define como una metodología que a través de una visión integral de la organización realiza la alineación de la filosofía organizacional con los diferentes procesos al interior de la empresa, identificando la interrelaciones entre los mismos, teniendo en cuenta la estructura de ADM (Architecture Development Method), permitiendo que los procesos estratégicos y de apoyo tengan un horizonte definido formalmente.

Un instrumento fundamental para el desarrollo del trabajo de grado fue Archimate, que es un lenguaje gráfico por medio del cual se representa la arquitectura empresarial que soporta la operación de la empresa, con esta herramienta se realizó la descripción de los diferentes puntos de vista que abarcan toda la operación generando la visión global de la misma y su respectiva alineación con los objetivos estratégicos.

{\large\textbf{Palabras Clave:}}
Archimate, Arquitectura, Patrones, Colosoft, Creatics

\vspace*{20mm}
{\usekomafont{chapter}Abstract}
\label{sec:abstract} \\
Hola 2