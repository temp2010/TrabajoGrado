\thispagestyle{empty}
%\hfill
\vspace*{3cm}
\noindent\Huge\textsc{Conclusiones}\\
\normalsize
\noindent\rule[2pt]{\textwidth}{0.8pt}
\hspace*{3cm}

Se evidenció que Archimate fue un lenguaje apropiado a las necesidades de identificación de factores y relaciones entre los mismos para el desarrollo de la arquitectura empresarial ya que a través de los diferentes puntos de vista que propone se logró abarcar todos los procesos y alinearlos con la filosofía organizacional. \\

El desarrollo de la Arquitectura Empresarial permitió identificar debilidades y fortalezas a nivel interno y oportunidades y amenazas a nivel externo teniendo en cuenta los diferentes stakeholders. \\

La alineación de todos los procesos con los objetivos estratégicos permite que las políticas y planes de acción generados en los procesos estratégicos y de apoyo tengan el mismo horizonte que los procesos misionales. \\

La arquitectura empresarial permite definir el estado ideal en el que se desea que se encuentre la organización, contribuyendo a la generación de una planeación estratégica con bases realmente sólidas que garantice que la organización cumple las expectativas y requerimientos de las instituciones educativas en cuando a necesidades de gestión y se encuentra en capacidad de responder acertadamente a los cambios que se presentan.