\thispagestyle{empty}
%\hfill
\vspace*{3cm}
\noindent\Huge\textsc{Bibliografia}\\
\normalsize
\noindent\rule[2pt]{\textwidth}{0.8pt}
\hspace*{3cm}

[1] Prada Sarmiento, Luis Felipe. El Universo LATEX Rodrigo De Castro Korgi. Universidad Nacional de Colombia. Segunda Edición. 2003.

FINALIDAD DEL COMPLEMENTO MÓDULOS
[2] S. J. Bolaños Castro and R. Gonzalez Crespo. A software architecture proposal artistic engineering environment -aee-. Engineering Applications (WEA),2012 Workshop on,	pages 1-6, 2012.

[3] S. J. Bolaños, R. G. Crespo, O. Sanjuan Martinez, J. P. Espada and V. H. Medina Garcia. Coloso: A development environment centered process and intent. In Information Systems and Technologies (CISTI), 2012 7th Iberian Conference on, pages 1-6 .IEEE Conference Publications, 2012

[5] Pressman, R. (2006). Ingenieria de Software, un enfoque practico. Mexico: Mc Graw Hill.
Programming, E. (s.f.). Rational Unified Process. Recuperado el 29 de 02 de 2014, de http://www.usmp.edu.pe/publicaciones/boletin/fia/info49/articulos/RUP%20vs.%20XP.pdf

[6] Módulo de Programación Orientada a Objetos, Apuntes de clase, Profesor Fernando Martínez Rodríguez, Universidad Distrital Francisco José de Caldas, 2015.

[7] Introducción a la Programación Orientada a Objetos, Luis R. Izquierdo, http://luis.izqui.org/resources/ProgOrientadaObjetos.pdf, 2015

[8] Módulo de Patrones, Apuntes de clase, Profesor Jhon Francined Herrera Cubides, Universidad Distrital Francisco José de Caldas, 2015.

[9] Somerville, I. (2005). Ingenieria de Software. España: Pearson Addison Wesley.
Ticona Condori, S. F. (s.f.). Metodologias tradicionales, metodologias agiles. Recuperado el 28 de 02 de 2014

TODO EL CAPITULO 3
[10] Open Group Stadard, Archimate 2.0 Specification – Language Structure, Business, Aplication, Technology Layers, Relationships, Motivation, Implementation and Migration Extension. Document Number C118. 2012

[11] Architecture Development Method (ADM), Introduction to the ADM
http://pubs.opengroup.org/architecture/togaf9-doc/arch/chap05.html


[12] EURACHEM. The Fitness for Purpose of Analytical Methods. A Laboratory Guide to Method Validation and Related Topics. EURACHEM Guide. 1998. Disponible en http://www.eurachem.ul.pt/. 

[13] Sistemas de gestión de la calidad – Principios y vocabulario. NMX-CC-9000-IMNC-2001. 2001

[14] Gonzalez, Rafael A. "Validation of Crisis Response Simulation within the Design Science Framework". ICIS 2009 Proceedings. Paper 87. 2009. http://aisel.aisnet.org/icis2009/87 

[15] Sistemas de gestión de la calidad – Requisitos. NMX-CC-9001-IMNC-2001. 2001 

[16] Majewski, M., Han, Q., Wurster, A. “Business Process Validation”. University of Augsburg. 2009 

[17] Khatri, V., Vessey, I., Ramesh, V., Clay, P., Park, S. “Understanding conceptual schemas: exploring the role of application and IS Domain knowledge”. Information Systems Research. Vol. 17, Number 1. Pp: 81-99. 2006. 

[18] Weber, I., Hoffmann, J., Mendling, J. “Semantic Business Process Validation”. In Proc. of International workshop on Semantic Business Process Management. 2008.